\capitulo{2}{Objetivos}

Este trabajo tiene como propósito principal desarrollar una red neuronal capaz de identificar correctamente la presencia o ausencia de neumonía en CXT de pacientes de entre 1 y 5 años. Esto también involucra la adquisición de conocimientos acerca de la neumonía y las CXT entre otros. 

\section{Objetivos generales}

\begin{itemize}
    \item Encontrar el mejor modelo de red neuronal capaz de identificar la presencia o ausencia de neumonía en imágenes de CXT.
    \item Investigar en profundidad distintos aspectos de la neumonía tales como sus causas, síntomas, diagnósticos y tratamientos, para contextualizar la necesidad de mejora en la identificación de neumonía a partir de imágenes de radiografía de tórax.
\end{itemize}

\section{Objetivos técnicos}

\begin{itemize}
    \item Desarrollar una red neuronal convolucional (CNN) propia y compararla con la CNN de AlexNet. 
    \item Determinar la arquitectura de red neuronal óptima (en precisión y eficiencia computacional) junto con el mejor tamaño de \textit{batch size} mediante un análisis comparativo. 
    \item Comparar y determinar el número óptimo de neuronas en la capa oculta para maximizar el rendimiento del modelo.
    \item Emplear el método \textit{hold-out} para dividir los datos en entrenamiento (train), validación (val) y prueba (test). De forma que, se utilizan parte de los datos para entrenar el modelo y la otra para validarlo y probarlo. Por ende, se consigue una mejor capacidad de generalización evitando que el modelo solo trabaje bien con datos conocidos.
\end{itemize}

\section{Objetivos personales}

\begin{itemize}
    \item Profundizar en el conocimiento de la fisiología y patología del sistema respiratorio, con especial atención a la neumonía.
    \item Recordar y afianzar viejos conocimientos acerca de LaTex y GitHub.
    \item Profundizar en el conocimiento de las CXT y el análisis de sus imágenes.
    \item Afianzar conocimientos acerca de las redes neuronales.
    \item Aprender a utilizar keras, una herramienta importante en la clasificación de imágenes a partir del aprendizaje profundo.
\end{itemize}























