\capitulo{7}{Lineas de trabajo futuras}

En este capítulo, se proponen diversas líneas de trabajo para continuar y mejorar el trabajo realizado.

\begin{itemize}
    \item Una línea futura para este trabajo consiste en realizar la anotación de cada una de las imágenes de radiografías de tórax con la ayuda de un médico especializado. De forma que, el profesional señale de forma precisa la ubicación de la neumonía y, así, la red neuronal puede entrenar las imágenes de una forma más precisa. Esta metodología, que se consideró inicialmente como una opción para este proyecto (en la segunda opción de TFG), no pudo llevarse a cabo en esta ocasión, pero puede tratarse de una gran mejora para futuros desarrollos.
    \item Otra línea futura puede ser, mejorar la red neuronal entrenada de forma que, no solo identifique la presencia de neumonía, sino que también, sea capaz de distinguir entre neumonía viral y bacteriana. Esta mejora, permitiría personalizar el tratamiento desde el primer momento, optimizando la atención médica y mejorando los resultados clínicos de los pacientes. Las imágenes con las que se trabaja en las carpetas de ``PNEUMONIA'' ya están calificadas como ``viral'' o ``bacteriana'' lo que simplifica el proceso.
    \item Una visión a futuro más lejano, pero con una gran relevancia, es la aplicación de estas técnicas en el ámbito clínico para determinar tanto la presencia de neumonía como su localización exacta en una CXT. De forma que, cuando un paciente acuda a consulta y se le realice una CXT, el sistema proporcione de forma automática al médico, información precisa sobre la existencia (o ausencia) y localización de la neumonía. Esta implementación optimizaría el diagnóstico y tratamiento y mejoraría la atención al paciente a nivel mundial.
    \item En este caso, se trabaja con imágenes de radiografías de tórax de pacientes de entre 1 y 5 años por lo que, a la hora de analizar las imágenes, esta red neuronal está limitada a un rango de edad especifico. En un futuro, lo ideal sería añadir más imágenes, entre ellas imágenes de adultos a este entrenamiento para poder analizar la existencia de neumonía sin importar la edad.
    \item Mostrar al médico a partir de la CXT información acerca de la anomalía encontrada, si se encuentra en el pulmón derecho o el izquierdo, tipo, tamaño, densidad, homogeneidad, etc. 
    \item Realizar otra red neuronal de forma similar a esta, pero, incluyendo imágenes de CXT desde otros puntos de vista (además de la vista frontal) para así, poder abordar los problemas asociados con los ``puntos ciegos'' que pueden surgir al utilizar únicamente imágenes frontales. Al incluir nuevas vistas (como la lateral u oblicua), se espera mejorar la precisión y la robustez del modelo.
\end{itemize}
